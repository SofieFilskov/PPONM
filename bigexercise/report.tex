\documentclass[a4paper]{article}
\usepackage{amsmath}
%%\usepackage{graphicx}

\title{Finding the cubic root}
\author{Sofie Filskov Hermansen}
\date{\today}

\begin{document}
\maketitle


\section{Introduction}
The aim of this report is to implement a function that calculates the cubic root of a real positive number $x$. This is found by solving the equation
\begin{align}
c^3-x = 0.
\end{align}

We have earlier used the Multidimensional Root-Finding routine from GSL to find the root of a multidimensional problem, but as this problem is only one dimensional, we instead use the One Dimensional Root-Finding routine. This can be found in the header file \textit{gsl\_roots.h}.

\section{Solving the problem}
First the function is defined. For this, \textit{gsl\_function} is used, which only takes the parameter $x$ and the value to solve $c$ and returns the function $c^3-x = 0$. 

The type is set to \textit{gsl\_root\_fsolver\_brent} and we initialize the solver state for this algorithm. We use root bracketing, where we give a bounded region, as these algorithms are guaranteed to converge, whereas root polishing (where you give an initial guess) only converge if the initial guess is close enough to the true value of a root.
The bounded region must contain a root, så we start with the bounded region going from $0$ to $x$. The size of the bounded region is then reduced, iteratively, until it encloses the root to a desired tolerance. 

The solver state is updated via an iteration and tested for convergence. If the solver state has not converged, it is again updated via an iteration and so forth.

The function has converged when the condition
\begin{align}
|a-b| < epsabs + epsrel \text{min}(|a|,|b|)
\end{align}
is achieved, where $a$ and $b$ are the high and lower boundary, respectively, $epsabs$ is the absolute error and $epsrel$ is the relative error.

In figure \ref{my_figure} one can see how the guess on the cubic root comes closer to the true value with every iteration, until it is close enough to have converged. The figure shows the calculation for $x = 100$, and it is clear that the first guess is at half of this, 50. 

\begin{figure}
	% GNUPLOT: LaTeX picture with Postscript
\begingroup%
\makeatletter%
\newcommand{\GNUPLOTspecial}{%
  \@sanitize\catcode`\%=14\relax\special}%
\setlength{\unitlength}{0.0500bp}%
\begin{picture}(7200,5040)(0,0)%
  {\GNUPLOTspecial{"
%!PS-Adobe-2.0 EPSF-2.0
%%Title: plot.tex
%%Creator: gnuplot 5.0 patchlevel 3
%%CreationDate: Sat Mar 31 22:24:06 2018
%%DocumentFonts: 
%%BoundingBox: 0 0 360 252
%%EndComments
%%BeginProlog
/gnudict 256 dict def
gnudict begin
%
% The following true/false flags may be edited by hand if desired.
% The unit line width and grayscale image gamma correction may also be changed.
%
/Color false def
/Blacktext true def
/Solid false def
/Dashlength 1 def
/Landscape false def
/Level1 false def
/Level3 false def
/Rounded false def
/ClipToBoundingBox false def
/SuppressPDFMark false def
/TransparentPatterns false def
/gnulinewidth 5.000 def
/userlinewidth gnulinewidth def
/Gamma 1.0 def
/BackgroundColor {-1.000 -1.000 -1.000} def
%
/vshift -66 def
/dl1 {
  10.0 Dashlength userlinewidth gnulinewidth div mul mul mul
  Rounded { currentlinewidth 0.75 mul sub dup 0 le { pop 0.01 } if } if
} def
/dl2 {
  10.0 Dashlength userlinewidth gnulinewidth div mul mul mul
  Rounded { currentlinewidth 0.75 mul add } if
} def
/hpt_ 31.5 def
/vpt_ 31.5 def
/hpt hpt_ def
/vpt vpt_ def
/doclip {
  ClipToBoundingBox {
    newpath 0 0 moveto 360 0 lineto 360 252 lineto 0 252 lineto closepath
    clip
  } if
} def
%
% Gnuplot Prolog Version 5.1 (Oct 2015)
%
%/SuppressPDFMark true def
%
/M {moveto} bind def
/L {lineto} bind def
/R {rmoveto} bind def
/V {rlineto} bind def
/N {newpath moveto} bind def
/Z {closepath} bind def
/C {setrgbcolor} bind def
/f {rlineto fill} bind def
/g {setgray} bind def
/Gshow {show} def   % May be redefined later in the file to support UTF-8
/vpt2 vpt 2 mul def
/hpt2 hpt 2 mul def
/Lshow {currentpoint stroke M 0 vshift R 
	Blacktext {gsave 0 setgray textshow grestore} {textshow} ifelse} def
/Rshow {currentpoint stroke M dup stringwidth pop neg vshift R
	Blacktext {gsave 0 setgray textshow grestore} {textshow} ifelse} def
/Cshow {currentpoint stroke M dup stringwidth pop -2 div vshift R 
	Blacktext {gsave 0 setgray textshow grestore} {textshow} ifelse} def
/UP {dup vpt_ mul /vpt exch def hpt_ mul /hpt exch def
  /hpt2 hpt 2 mul def /vpt2 vpt 2 mul def} def
/DL {Color {setrgbcolor Solid {pop []} if 0 setdash}
 {pop pop pop 0 setgray Solid {pop []} if 0 setdash} ifelse} def
/BL {stroke userlinewidth 2 mul setlinewidth
	Rounded {1 setlinejoin 1 setlinecap} if} def
/AL {stroke userlinewidth 2 div setlinewidth
	Rounded {1 setlinejoin 1 setlinecap} if} def
/UL {dup gnulinewidth mul /userlinewidth exch def
	dup 1 lt {pop 1} if 10 mul /udl exch def} def
/PL {stroke userlinewidth setlinewidth
	Rounded {1 setlinejoin 1 setlinecap} if} def
3.8 setmiterlimit
% Classic Line colors (version 5.0)
/LCw {1 1 1} def
/LCb {0 0 0} def
/LCa {0 0 0} def
/LC0 {1 0 0} def
/LC1 {0 1 0} def
/LC2 {0 0 1} def
/LC3 {1 0 1} def
/LC4 {0 1 1} def
/LC5 {1 1 0} def
/LC6 {0 0 0} def
/LC7 {1 0.3 0} def
/LC8 {0.5 0.5 0.5} def
% Default dash patterns (version 5.0)
/LTB {BL [] LCb DL} def
/LTw {PL [] 1 setgray} def
/LTb {PL [] LCb DL} def
/LTa {AL [1 udl mul 2 udl mul] 0 setdash LCa setrgbcolor} def
/LT0 {PL [] LC0 DL} def
/LT1 {PL [2 dl1 3 dl2] LC1 DL} def
/LT2 {PL [1 dl1 1.5 dl2] LC2 DL} def
/LT3 {PL [6 dl1 2 dl2 1 dl1 2 dl2] LC3 DL} def
/LT4 {PL [1 dl1 2 dl2 6 dl1 2 dl2 1 dl1 2 dl2] LC4 DL} def
/LT5 {PL [4 dl1 2 dl2] LC5 DL} def
/LT6 {PL [1.5 dl1 1.5 dl2 1.5 dl1 1.5 dl2 1.5 dl1 6 dl2] LC6 DL} def
/LT7 {PL [3 dl1 3 dl2 1 dl1 3 dl2] LC7 DL} def
/LT8 {PL [2 dl1 2 dl2 2 dl1 6 dl2] LC8 DL} def
/SL {[] 0 setdash} def
/Pnt {stroke [] 0 setdash gsave 1 setlinecap M 0 0 V stroke grestore} def
/Dia {stroke [] 0 setdash 2 copy vpt add M
  hpt neg vpt neg V hpt vpt neg V
  hpt vpt V hpt neg vpt V closepath stroke
  Pnt} def
/Pls {stroke [] 0 setdash vpt sub M 0 vpt2 V
  currentpoint stroke M
  hpt neg vpt neg R hpt2 0 V stroke
 } def
/Box {stroke [] 0 setdash 2 copy exch hpt sub exch vpt add M
  0 vpt2 neg V hpt2 0 V 0 vpt2 V
  hpt2 neg 0 V closepath stroke
  Pnt} def
/Crs {stroke [] 0 setdash exch hpt sub exch vpt add M
  hpt2 vpt2 neg V currentpoint stroke M
  hpt2 neg 0 R hpt2 vpt2 V stroke} def
/TriU {stroke [] 0 setdash 2 copy vpt 1.12 mul add M
  hpt neg vpt -1.62 mul V
  hpt 2 mul 0 V
  hpt neg vpt 1.62 mul V closepath stroke
  Pnt} def
/Star {2 copy Pls Crs} def
/BoxF {stroke [] 0 setdash exch hpt sub exch vpt add M
  0 vpt2 neg V hpt2 0 V 0 vpt2 V
  hpt2 neg 0 V closepath fill} def
/TriUF {stroke [] 0 setdash vpt 1.12 mul add M
  hpt neg vpt -1.62 mul V
  hpt 2 mul 0 V
  hpt neg vpt 1.62 mul V closepath fill} def
/TriD {stroke [] 0 setdash 2 copy vpt 1.12 mul sub M
  hpt neg vpt 1.62 mul V
  hpt 2 mul 0 V
  hpt neg vpt -1.62 mul V closepath stroke
  Pnt} def
/TriDF {stroke [] 0 setdash vpt 1.12 mul sub M
  hpt neg vpt 1.62 mul V
  hpt 2 mul 0 V
  hpt neg vpt -1.62 mul V closepath fill} def
/DiaF {stroke [] 0 setdash vpt add M
  hpt neg vpt neg V hpt vpt neg V
  hpt vpt V hpt neg vpt V closepath fill} def
/Pent {stroke [] 0 setdash 2 copy gsave
  translate 0 hpt M 4 {72 rotate 0 hpt L} repeat
  closepath stroke grestore Pnt} def
/PentF {stroke [] 0 setdash gsave
  translate 0 hpt M 4 {72 rotate 0 hpt L} repeat
  closepath fill grestore} def
/Circle {stroke [] 0 setdash 2 copy
  hpt 0 360 arc stroke Pnt} def
/CircleF {stroke [] 0 setdash hpt 0 360 arc fill} def
/C0 {BL [] 0 setdash 2 copy moveto vpt 90 450 arc} bind def
/C1 {BL [] 0 setdash 2 copy moveto
	2 copy vpt 0 90 arc closepath fill
	vpt 0 360 arc closepath} bind def
/C2 {BL [] 0 setdash 2 copy moveto
	2 copy vpt 90 180 arc closepath fill
	vpt 0 360 arc closepath} bind def
/C3 {BL [] 0 setdash 2 copy moveto
	2 copy vpt 0 180 arc closepath fill
	vpt 0 360 arc closepath} bind def
/C4 {BL [] 0 setdash 2 copy moveto
	2 copy vpt 180 270 arc closepath fill
	vpt 0 360 arc closepath} bind def
/C5 {BL [] 0 setdash 2 copy moveto
	2 copy vpt 0 90 arc
	2 copy moveto
	2 copy vpt 180 270 arc closepath fill
	vpt 0 360 arc} bind def
/C6 {BL [] 0 setdash 2 copy moveto
	2 copy vpt 90 270 arc closepath fill
	vpt 0 360 arc closepath} bind def
/C7 {BL [] 0 setdash 2 copy moveto
	2 copy vpt 0 270 arc closepath fill
	vpt 0 360 arc closepath} bind def
/C8 {BL [] 0 setdash 2 copy moveto
	2 copy vpt 270 360 arc closepath fill
	vpt 0 360 arc closepath} bind def
/C9 {BL [] 0 setdash 2 copy moveto
	2 copy vpt 270 450 arc closepath fill
	vpt 0 360 arc closepath} bind def
/C10 {BL [] 0 setdash 2 copy 2 copy moveto vpt 270 360 arc closepath fill
	2 copy moveto
	2 copy vpt 90 180 arc closepath fill
	vpt 0 360 arc closepath} bind def
/C11 {BL [] 0 setdash 2 copy moveto
	2 copy vpt 0 180 arc closepath fill
	2 copy moveto
	2 copy vpt 270 360 arc closepath fill
	vpt 0 360 arc closepath} bind def
/C12 {BL [] 0 setdash 2 copy moveto
	2 copy vpt 180 360 arc closepath fill
	vpt 0 360 arc closepath} bind def
/C13 {BL [] 0 setdash 2 copy moveto
	2 copy vpt 0 90 arc closepath fill
	2 copy moveto
	2 copy vpt 180 360 arc closepath fill
	vpt 0 360 arc closepath} bind def
/C14 {BL [] 0 setdash 2 copy moveto
	2 copy vpt 90 360 arc closepath fill
	vpt 0 360 arc} bind def
/C15 {BL [] 0 setdash 2 copy vpt 0 360 arc closepath fill
	vpt 0 360 arc closepath} bind def
/Rec {newpath 4 2 roll moveto 1 index 0 rlineto 0 exch rlineto
	neg 0 rlineto closepath} bind def
/Square {dup Rec} bind def
/Bsquare {vpt sub exch vpt sub exch vpt2 Square} bind def
/S0 {BL [] 0 setdash 2 copy moveto 0 vpt rlineto BL Bsquare} bind def
/S1 {BL [] 0 setdash 2 copy vpt Square fill Bsquare} bind def
/S2 {BL [] 0 setdash 2 copy exch vpt sub exch vpt Square fill Bsquare} bind def
/S3 {BL [] 0 setdash 2 copy exch vpt sub exch vpt2 vpt Rec fill Bsquare} bind def
/S4 {BL [] 0 setdash 2 copy exch vpt sub exch vpt sub vpt Square fill Bsquare} bind def
/S5 {BL [] 0 setdash 2 copy 2 copy vpt Square fill
	exch vpt sub exch vpt sub vpt Square fill Bsquare} bind def
/S6 {BL [] 0 setdash 2 copy exch vpt sub exch vpt sub vpt vpt2 Rec fill Bsquare} bind def
/S7 {BL [] 0 setdash 2 copy exch vpt sub exch vpt sub vpt vpt2 Rec fill
	2 copy vpt Square fill Bsquare} bind def
/S8 {BL [] 0 setdash 2 copy vpt sub vpt Square fill Bsquare} bind def
/S9 {BL [] 0 setdash 2 copy vpt sub vpt vpt2 Rec fill Bsquare} bind def
/S10 {BL [] 0 setdash 2 copy vpt sub vpt Square fill 2 copy exch vpt sub exch vpt Square fill
	Bsquare} bind def
/S11 {BL [] 0 setdash 2 copy vpt sub vpt Square fill 2 copy exch vpt sub exch vpt2 vpt Rec fill
	Bsquare} bind def
/S12 {BL [] 0 setdash 2 copy exch vpt sub exch vpt sub vpt2 vpt Rec fill Bsquare} bind def
/S13 {BL [] 0 setdash 2 copy exch vpt sub exch vpt sub vpt2 vpt Rec fill
	2 copy vpt Square fill Bsquare} bind def
/S14 {BL [] 0 setdash 2 copy exch vpt sub exch vpt sub vpt2 vpt Rec fill
	2 copy exch vpt sub exch vpt Square fill Bsquare} bind def
/S15 {BL [] 0 setdash 2 copy Bsquare fill Bsquare} bind def
/D0 {gsave translate 45 rotate 0 0 S0 stroke grestore} bind def
/D1 {gsave translate 45 rotate 0 0 S1 stroke grestore} bind def
/D2 {gsave translate 45 rotate 0 0 S2 stroke grestore} bind def
/D3 {gsave translate 45 rotate 0 0 S3 stroke grestore} bind def
/D4 {gsave translate 45 rotate 0 0 S4 stroke grestore} bind def
/D5 {gsave translate 45 rotate 0 0 S5 stroke grestore} bind def
/D6 {gsave translate 45 rotate 0 0 S6 stroke grestore} bind def
/D7 {gsave translate 45 rotate 0 0 S7 stroke grestore} bind def
/D8 {gsave translate 45 rotate 0 0 S8 stroke grestore} bind def
/D9 {gsave translate 45 rotate 0 0 S9 stroke grestore} bind def
/D10 {gsave translate 45 rotate 0 0 S10 stroke grestore} bind def
/D11 {gsave translate 45 rotate 0 0 S11 stroke grestore} bind def
/D12 {gsave translate 45 rotate 0 0 S12 stroke grestore} bind def
/D13 {gsave translate 45 rotate 0 0 S13 stroke grestore} bind def
/D14 {gsave translate 45 rotate 0 0 S14 stroke grestore} bind def
/D15 {gsave translate 45 rotate 0 0 S15 stroke grestore} bind def
/DiaE {stroke [] 0 setdash vpt add M
  hpt neg vpt neg V hpt vpt neg V
  hpt vpt V hpt neg vpt V closepath stroke} def
/BoxE {stroke [] 0 setdash exch hpt sub exch vpt add M
  0 vpt2 neg V hpt2 0 V 0 vpt2 V
  hpt2 neg 0 V closepath stroke} def
/TriUE {stroke [] 0 setdash vpt 1.12 mul add M
  hpt neg vpt -1.62 mul V
  hpt 2 mul 0 V
  hpt neg vpt 1.62 mul V closepath stroke} def
/TriDE {stroke [] 0 setdash vpt 1.12 mul sub M
  hpt neg vpt 1.62 mul V
  hpt 2 mul 0 V
  hpt neg vpt -1.62 mul V closepath stroke} def
/PentE {stroke [] 0 setdash gsave
  translate 0 hpt M 4 {72 rotate 0 hpt L} repeat
  closepath stroke grestore} def
/CircE {stroke [] 0 setdash 
  hpt 0 360 arc stroke} def
/Opaque {gsave closepath 1 setgray fill grestore 0 setgray closepath} def
/DiaW {stroke [] 0 setdash vpt add M
  hpt neg vpt neg V hpt vpt neg V
  hpt vpt V hpt neg vpt V Opaque stroke} def
/BoxW {stroke [] 0 setdash exch hpt sub exch vpt add M
  0 vpt2 neg V hpt2 0 V 0 vpt2 V
  hpt2 neg 0 V Opaque stroke} def
/TriUW {stroke [] 0 setdash vpt 1.12 mul add M
  hpt neg vpt -1.62 mul V
  hpt 2 mul 0 V
  hpt neg vpt 1.62 mul V Opaque stroke} def
/TriDW {stroke [] 0 setdash vpt 1.12 mul sub M
  hpt neg vpt 1.62 mul V
  hpt 2 mul 0 V
  hpt neg vpt -1.62 mul V Opaque stroke} def
/PentW {stroke [] 0 setdash gsave
  translate 0 hpt M 4 {72 rotate 0 hpt L} repeat
  Opaque stroke grestore} def
/CircW {stroke [] 0 setdash 
  hpt 0 360 arc Opaque stroke} def
/BoxFill {gsave Rec 1 setgray fill grestore} def
/Density {
  /Fillden exch def
  currentrgbcolor
  /ColB exch def /ColG exch def /ColR exch def
  /ColR ColR Fillden mul Fillden sub 1 add def
  /ColG ColG Fillden mul Fillden sub 1 add def
  /ColB ColB Fillden mul Fillden sub 1 add def
  ColR ColG ColB setrgbcolor} def
/BoxColFill {gsave Rec PolyFill} def
/PolyFill {gsave Density fill grestore grestore} def
/h {rlineto rlineto rlineto gsave closepath fill grestore} bind def
%
% PostScript Level 1 Pattern Fill routine for rectangles
% Usage: x y w h s a XX PatternFill
%	x,y = lower left corner of box to be filled
%	w,h = width and height of box
%	  a = angle in degrees between lines and x-axis
%	 XX = 0/1 for no/yes cross-hatch
%
/PatternFill {gsave /PFa [ 9 2 roll ] def
  PFa 0 get PFa 2 get 2 div add PFa 1 get PFa 3 get 2 div add translate
  PFa 2 get -2 div PFa 3 get -2 div PFa 2 get PFa 3 get Rec
  TransparentPatterns {} {gsave 1 setgray fill grestore} ifelse
  clip
  currentlinewidth 0.5 mul setlinewidth
  /PFs PFa 2 get dup mul PFa 3 get dup mul add sqrt def
  0 0 M PFa 5 get rotate PFs -2 div dup translate
  0 1 PFs PFa 4 get div 1 add floor cvi
	{PFa 4 get mul 0 M 0 PFs V} for
  0 PFa 6 get ne {
	0 1 PFs PFa 4 get div 1 add floor cvi
	{PFa 4 get mul 0 2 1 roll M PFs 0 V} for
 } if
  stroke grestore} def
%
/languagelevel where
 {pop languagelevel} {1} ifelse
dup 2 lt
	{/InterpretLevel1 true def
	 /InterpretLevel3 false def}
	{/InterpretLevel1 Level1 def
	 2 gt
	    {/InterpretLevel3 Level3 def}
	    {/InterpretLevel3 false def}
	 ifelse }
 ifelse
%
% PostScript level 2 pattern fill definitions
%
/Level2PatternFill {
/Tile8x8 {/PaintType 2 /PatternType 1 /TilingType 1 /BBox [0 0 8 8] /XStep 8 /YStep 8}
	bind def
/KeepColor {currentrgbcolor [/Pattern /DeviceRGB] setcolorspace} bind def
<< Tile8x8
 /PaintProc {0.5 setlinewidth pop 0 0 M 8 8 L 0 8 M 8 0 L stroke} 
>> matrix makepattern
/Pat1 exch def
<< Tile8x8
 /PaintProc {0.5 setlinewidth pop 0 0 M 8 8 L 0 8 M 8 0 L stroke
	0 4 M 4 8 L 8 4 L 4 0 L 0 4 L stroke}
>> matrix makepattern
/Pat2 exch def
<< Tile8x8
 /PaintProc {0.5 setlinewidth pop 0 0 M 0 8 L
	8 8 L 8 0 L 0 0 L fill}
>> matrix makepattern
/Pat3 exch def
<< Tile8x8
 /PaintProc {0.5 setlinewidth pop -4 8 M 8 -4 L
	0 12 M 12 0 L stroke}
>> matrix makepattern
/Pat4 exch def
<< Tile8x8
 /PaintProc {0.5 setlinewidth pop -4 0 M 8 12 L
	0 -4 M 12 8 L stroke}
>> matrix makepattern
/Pat5 exch def
<< Tile8x8
 /PaintProc {0.5 setlinewidth pop -2 8 M 4 -4 L
	0 12 M 8 -4 L 4 12 M 10 0 L stroke}
>> matrix makepattern
/Pat6 exch def
<< Tile8x8
 /PaintProc {0.5 setlinewidth pop -2 0 M 4 12 L
	0 -4 M 8 12 L 4 -4 M 10 8 L stroke}
>> matrix makepattern
/Pat7 exch def
<< Tile8x8
 /PaintProc {0.5 setlinewidth pop 8 -2 M -4 4 L
	12 0 M -4 8 L 12 4 M 0 10 L stroke}
>> matrix makepattern
/Pat8 exch def
<< Tile8x8
 /PaintProc {0.5 setlinewidth pop 0 -2 M 12 4 L
	-4 0 M 12 8 L -4 4 M 8 10 L stroke}
>> matrix makepattern
/Pat9 exch def
/Pattern1 {PatternBgnd KeepColor Pat1 setpattern} bind def
/Pattern2 {PatternBgnd KeepColor Pat2 setpattern} bind def
/Pattern3 {PatternBgnd KeepColor Pat3 setpattern} bind def
/Pattern4 {PatternBgnd KeepColor Landscape {Pat5} {Pat4} ifelse setpattern} bind def
/Pattern5 {PatternBgnd KeepColor Landscape {Pat4} {Pat5} ifelse setpattern} bind def
/Pattern6 {PatternBgnd KeepColor Landscape {Pat9} {Pat6} ifelse setpattern} bind def
/Pattern7 {PatternBgnd KeepColor Landscape {Pat8} {Pat7} ifelse setpattern} bind def
} def
%
%
%End of PostScript Level 2 code
%
/PatternBgnd {
  TransparentPatterns {} {gsave 1 setgray fill grestore} ifelse
} def
%
% Substitute for Level 2 pattern fill codes with
% grayscale if Level 2 support is not selected.
%
/Level1PatternFill {
/Pattern1 {0.250 Density} bind def
/Pattern2 {0.500 Density} bind def
/Pattern3 {0.750 Density} bind def
/Pattern4 {0.125 Density} bind def
/Pattern5 {0.375 Density} bind def
/Pattern6 {0.625 Density} bind def
/Pattern7 {0.875 Density} bind def
} def
%
% Now test for support of Level 2 code
%
Level1 {Level1PatternFill} {Level2PatternFill} ifelse
%
/Symbol-Oblique /Symbol findfont [1 0 .167 1 0 0] makefont
dup length dict begin {1 index /FID eq {pop pop} {def} ifelse} forall
currentdict end definefont pop
%
Level1 SuppressPDFMark or 
{} {
/SDict 10 dict def
systemdict /pdfmark known not {
  userdict /pdfmark systemdict /cleartomark get put
} if
SDict begin [
  /Title (plot.tex)
  /Subject (gnuplot plot)
  /Creator (gnuplot 5.0 patchlevel 3)
  /Author (sofie)
%  /Producer (gnuplot)
%  /Keywords ()
  /CreationDate (Sat Mar 31 22:24:06 2018)
  /DOCINFO pdfmark
end
} ifelse
%
% Support for boxed text - Ethan A Merritt May 2005
%
/InitTextBox { userdict /TBy2 3 -1 roll put userdict /TBx2 3 -1 roll put
           userdict /TBy1 3 -1 roll put userdict /TBx1 3 -1 roll put
	   /Boxing true def } def
/ExtendTextBox { Boxing
    { gsave dup false charpath pathbbox
      dup TBy2 gt {userdict /TBy2 3 -1 roll put} {pop} ifelse
      dup TBx2 gt {userdict /TBx2 3 -1 roll put} {pop} ifelse
      dup TBy1 lt {userdict /TBy1 3 -1 roll put} {pop} ifelse
      dup TBx1 lt {userdict /TBx1 3 -1 roll put} {pop} ifelse
      grestore } if } def
/PopTextBox { newpath TBx1 TBxmargin sub TBy1 TBymargin sub M
               TBx1 TBxmargin sub TBy2 TBymargin add L
	       TBx2 TBxmargin add TBy2 TBymargin add L
	       TBx2 TBxmargin add TBy1 TBymargin sub L closepath } def
/DrawTextBox { PopTextBox stroke /Boxing false def} def
/FillTextBox { gsave PopTextBox 1 1 1 setrgbcolor fill grestore /Boxing false def} def
0 0 0 0 InitTextBox
/TBxmargin 20 def
/TBymargin 20 def
/Boxing false def
/textshow { ExtendTextBox Gshow } def
%
% redundant definitions for compatibility with prologue.ps older than 5.0.2
/LTB {BL [] LCb DL} def
/LTb {PL [] LCb DL} def
end
%%EndProlog
%%Page: 1 1
gnudict begin
gsave
doclip
0 0 translate
0.050 0.050 scale
0 setgray
newpath
BackgroundColor 0 lt 3 1 roll 0 lt exch 0 lt or or not {BackgroundColor C 1.000 0 0 7200.00 5040.00 BoxColFill} if
1.000 UL
LTb
LCb setrgbcolor
740 640 M
63 0 V
6036 0 R
-63 0 V
stroke
LTb
LCb setrgbcolor
740 1018 M
63 0 V
6036 0 R
-63 0 V
stroke
LTb
LCb setrgbcolor
740 1396 M
63 0 V
6036 0 R
-63 0 V
stroke
LTb
LCb setrgbcolor
740 1774 M
63 0 V
6036 0 R
-63 0 V
stroke
LTb
LCb setrgbcolor
740 2152 M
63 0 V
6036 0 R
-63 0 V
stroke
LTb
LCb setrgbcolor
740 2530 M
63 0 V
6036 0 R
-63 0 V
stroke
LTb
LCb setrgbcolor
740 2909 M
63 0 V
6036 0 R
-63 0 V
stroke
LTb
LCb setrgbcolor
740 3287 M
63 0 V
6036 0 R
-63 0 V
stroke
LTb
LCb setrgbcolor
740 3665 M
63 0 V
6036 0 R
-63 0 V
stroke
LTb
LCb setrgbcolor
740 4043 M
63 0 V
6036 0 R
-63 0 V
stroke
LTb
LCb setrgbcolor
740 4421 M
63 0 V
6036 0 R
-63 0 V
stroke
LTb
LCb setrgbcolor
740 4799 M
63 0 V
6036 0 R
-63 0 V
stroke
LTb
LCb setrgbcolor
740 640 M
0 63 V
0 4096 R
0 -63 V
stroke
LTb
LCb setrgbcolor
1611 640 M
0 63 V
0 4096 R
0 -63 V
stroke
LTb
LCb setrgbcolor
2483 640 M
0 63 V
0 4096 R
0 -63 V
stroke
LTb
LCb setrgbcolor
3354 640 M
0 63 V
0 4096 R
0 -63 V
stroke
LTb
LCb setrgbcolor
4225 640 M
0 63 V
0 4096 R
0 -63 V
stroke
LTb
LCb setrgbcolor
5096 640 M
0 63 V
0 4096 R
0 -63 V
stroke
LTb
LCb setrgbcolor
5968 640 M
0 63 V
0 4096 R
0 -63 V
stroke
LTb
LCb setrgbcolor
6839 640 M
0 63 V
0 4096 R
0 -63 V
stroke
LTb
LCb setrgbcolor
1.000 UL
LTb
LCb setrgbcolor
740 4799 N
740 640 L
6099 0 V
0 4159 V
-6099 0 V
Z stroke
1.000 UP
1.000 UL
LTb
LCb setrgbcolor
LCb setrgbcolor
LTb
LCb setrgbcolor
LTb
1.000 UL
LTb
0.58 0.00 0.83 C LCb setrgbcolor
1.000 UL
LTb
0.58 0.00 0.83 C 6056 4636 M
543 0 V
740 4421 M
1176 641 L
435 3780 V
2047 644 L
436 1889 V
2918 656 L
436 938 V
3790 703 L
435 445 V
4661 849 L
435 150 V
436 -12 V
436 4 V
435 0 V
436 0 V
stroke
LTb
0.00 0.62 0.45 C LCb setrgbcolor
1.000 UL
LTb
0.00 0.62 0.45 C 6056 4436 M
543 0 V
740 991 M
436 0 V
435 0 V
436 0 V
436 0 V
435 0 V
436 0 V
436 0 V
435 0 V
436 0 V
435 0 V
436 0 V
436 0 V
435 0 V
436 0 V
stroke
2.000 UL
LTb
LCb setrgbcolor
1.000 UL
LTb
LCb setrgbcolor
740 4799 N
740 640 L
6099 0 V
0 4159 V
-6099 0 V
Z stroke
1.000 UP
1.000 UL
LTb
LCb setrgbcolor
stroke
grestore
end
showpage
  }}%
  \put(5936,4436){\makebox(0,0)[r]{\strut{}Expected value}}%
  \put(5936,4636){\makebox(0,0)[r]{\strut{}Cubic root finding}}%
  \put(3789,140){\makebox(0,0){\strut{}Iteration}}%
  \put(160,2719){%
  \special{ps: gsave currentpoint currentpoint translate
630 rotate neg exch neg exch translate}%
  \makebox(0,0){\strut{}Root}%
  \special{ps: currentpoint grestore moveto}%
  }%
  \put(6839,440){\makebox(0,0){\strut{}$14$}}%
  \put(5968,440){\makebox(0,0){\strut{}$12$}}%
  \put(5096,440){\makebox(0,0){\strut{}$10$}}%
  \put(4225,440){\makebox(0,0){\strut{}$8$}}%
  \put(3354,440){\makebox(0,0){\strut{}$6$}}%
  \put(2483,440){\makebox(0,0){\strut{}$4$}}%
  \put(1611,440){\makebox(0,0){\strut{}$2$}}%
  \put(740,440){\makebox(0,0){\strut{}$0$}}%
  \put(620,4799){\makebox(0,0)[r]{\strut{}$55$}}%
  \put(620,4421){\makebox(0,0)[r]{\strut{}$50$}}%
  \put(620,4043){\makebox(0,0)[r]{\strut{}$45$}}%
  \put(620,3665){\makebox(0,0)[r]{\strut{}$40$}}%
  \put(620,3287){\makebox(0,0)[r]{\strut{}$35$}}%
  \put(620,2909){\makebox(0,0)[r]{\strut{}$30$}}%
  \put(620,2530){\makebox(0,0)[r]{\strut{}$25$}}%
  \put(620,2152){\makebox(0,0)[r]{\strut{}$20$}}%
  \put(620,1774){\makebox(0,0)[r]{\strut{}$15$}}%
  \put(620,1396){\makebox(0,0)[r]{\strut{}$10$}}%
  \put(620,1018){\makebox(0,0)[r]{\strut{}$5$}}%
  \put(620,640){\makebox(0,0)[r]{\strut{}$0$}}%
\end{picture}%
\endgroup
\endinput

	\caption{A plot of the guess on the root versus iterations. Here, the parameter $x = 100$ and the cubic root is calculated to $c = 4.6415888$.}
	\label{my_figure}
\end{figure}

\end{document}
